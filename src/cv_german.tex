%%%%%%%%%%%%%%%%%%%%%%%%%%%%%%%%%%%%%%%%%
% Twenty Seconds Resume/CV
% LaTeX Template
% Version 1.0 (14/7/16)
%
% Original author:
% Carmine Spagnuolo (cspagnuolo@unisa.it) with major modifications by 
% Vel (vel@LaTeXTemplates.com), Harsh (harsh.gadgil@gmail.com)
% and ZabuzaW (zabuza.dev@gmail.com)
%
%%%%%%%%%%%%%%%%%%%%%%%%%%%%%%%%%%%%%%%%%

%----------------------------------------------------------------------------------------
%	PACKAGES AND OTHER DOCUMENT CONFIGURATIONS
%----------------------------------------------------------------------------------------

\documentclass[letterpaper]{twentysecondcv_german} % a4paper for A4

% Command for printing skill overview bubbles
\newcommand\skills{ 
~
	\smartdiagram[bubble diagram]{
		\textbf{Software}\\\textbf{Engineering},
		\textbf{Algorithm}\\\textbf{~~Engineering~~},
		\textbf{Web}\\\textbf{~~~~~~Dev~~~~~~},
		\textbf{Theoretical}\\\textbf{CS},
		\textbf{~~~Game~~~}\\\textbf{Dev},
		\textbf{Mathematical}\\\textbf{Analysis},
		\textbf{Agile}\\\textbf{Scrum}
	}
}

% Programming skill bars
\programming{{C $\textbullet$ C++  $\textbullet$ C\# $\textbullet$ Python / 2.1}, {PHP $\textbullet$ SQL $\textbullet$ HTML $\textbullet$ JS $\textbullet$ CSS / 3.3}, {Java / 5.7}}

% Command for printing language overview bubbles
\newcommand\languages{ 
~
	\smartdiagram[bubble diagram]{
		\textbf{~~~~~Deutsch~~~~~}\\\textbf{C2},
		\textbf{~~~~~~~~Englisch~~~~~~~~}\\\textbf{C1},
		\textbf{Chinesisch}\\\textbf{A2}
	}
}

%----------------------------------------------------------------------------------------
%	 PERSONAL INFORMATION
%----------------------------------------------------------------------------------------
% If you don't need one or more of the below, just remove the content leaving the command, e.g. \cvnumberphone{}

\cvname{DANIEL TISCHNER} % Your name
\cvjobtitle{ Softwareentwickler, 26 } % Job
% title/career

\cvlinkedin{/in/daniel-tischner} % LinkedIn site
\cvgithub{ZabuzaW} % Github site
\cvsite{} % Personal website
\cvnumberphone{+49 2132 5810229} % Phone number
\cvstackoverflow{Zabuza} % StackOverflow user name
\cvstackoverflowId{2411243} % StackOverflow user id
\cvstackoverflowRep{10,735} % StackOverflow user reputation
\cvstackoverflowGold{5} % StackOverflow amount of gold badges
\cvstackoverflowSilver{25} % StackOverflow amount of silver badges
\cvstackoverflowBronze{38} % StackOverflow amount of bronze badges
\cvmail{daniel.tischner.cs@gmail.com} % Email address

%----------------------------------------------------------------------------------------

\begin{document}

\makeprofile % Print the sidebar

%----------------------------------------------------------------------------------------
%	 EDUCATION
%----------------------------------------------------------------------------------------
\section{Bildung}

\begin{twenty} % Environment for a list with descriptions
	\twentyitem
    	{2016 - Heute~~~~}
        {}
        {M.~Sc., Informatik}
        {\href{https://www.uni-freiburg.de/}{Universität Freiburg, Deutschland}}
        {}
        {}
	\twentyitem
    	{2012 - 2016}
		{}
        {B.~~Sc., Informatik \textnormal{(GPA: 1.9)}}
        {\href{https://www.uni-freiburg.de/}{Universität Freiburg, Deutschland}}
        {}
        {}
	%\twentyitem{<dates>}{<title>}{<organization>}{<location>}{<description>}
\end{twenty}

%----------------------------------------------------------------------------------------
%	 EXPERIENCE
%----------------------------------------------------------------------------------------

\section{Erfahrung}

\begin{twenty} % Environment for a list with descriptions
     	\twentyitem
    		{Okt 2016 -}
		{Heute}
        		{Tutor}
        		{\href{https://www.uni-freiburg.de/}{Universität Freiburg, Deutschland}}
        		{}
        		{\begin{itemize}
        			\item Unterstützung von Studenten beim Verstehen der Kursinhalte, Organisation von wöchentlichen Meetings
        			\item Kurse:
        			\begin{itemize}
        				\item \textit{Programming in C++}
        				\item \textit{Information Retrieval}
        				\item \textit{Algorithms and data structures}
        				\item \textit{System Design Project}
        			\end{itemize}
        		\end{itemize}}\\
     	\twentyitem
    		{Nov 2008 -}
		{Heute}
        		{Administrator \& Content Creator}
        		{\href{http://www.fwwiki.de/}{FreewarWiki, Online}}
        		{}
        		{\begin{itemize}
        			\item Entwicklung von automatisierten Bots and Skripten, welche wiederkehrende Aufgaben ausführen;
        			Datenbankmanagement und Erstellung von gescripteten Vorlagen um unerfahrenen Nutzern die
        			Inhaltserstellung zu erleichtern
        			\item Umfangreiche Dokumentation von Spiel-Elementen und Inhaltserstellung
        			\item Sichtung, Korrektur und Verbesserung von Beiträgen
        		\end{itemize}}\\
	\twentyitem
    		{Nov 2014 -}
		{Dez 2015}
        		{Board \& Webadmin}
        		{\href{https://www.gruppe-w.de/}{Gruppe W, Online}}
        		{}
        		{\begin{itemize}
        			\item Web development, Perfomance Verbesserung, Entwicklung einer automatisierten web-integrierten
        			Managementsoftware, um den Arbeitsfluss der Mitglieder zu erleichtern
        			\item Personalmanagement, Projektmanagement, Planung und Halten von Konferenzen, Mediator
        			\item Entwicklung von Modifikationen für ArmA 3, Erstellung von 2D Artwork und Videobearbeitung
        		\end{itemize}}\\
	\twentyitem
    		{Mär 2007 -}
		{Jul~\, 2012}
        		{Lead Programmer \& Artist}
        		{\href{https://github.com/ZabuzaW/TalesOfFreewar}{Tales of Freewar Team, Deutschland}}
        		{}
        		{\begin{itemize}
        			\item Programmierung eines RPGs mit dem Spieleframework RM2k3; verantwortlich für
        			Spielelogik, Story, Level, Konversationen und Kampflogik.
        			item Erstellung von 2D Art und Soundbearbeitung, Design und Erstellung von Karten
        		\end{itemize}}\\
	%\twentyitem{<dates>}{<title>}{<location>}{<description>}
\end{twenty}

%----------------------------------------------------------------------------------------
%	 Awards
%----------------------------------------------------------------------------------------

\section{Auszeichnungen}
\begin{twenty} % Environment for a list with descriptions
	\twentyitem
    		{Jan 2017}
		{}
        		{SV-COMP 2017: Gold in Overall}
        		{\href{https://sv-comp.sosy-lab.org/2017/}{ETAPS/SV-COMP}}
        		{}
        		{\begin{itemize}
        			\item Der Wettbewerb vergleicht state-of-the-art Tools für Software Verifizierung
        			in Bezug auf Effektivität und Effizienz. Der Wettbewerb besteht aus zwei Phasen:
        			Eine Trainings-Phase und eine Evaluations-Phase.
        		\end{itemize}}\\
        	\twentyitem
    		{Okt 2016}
		{}
        		{RERS Challenge 2016: Gold in Overall}
        		{\href{http://rers-challenge.org/2016/}{RERS / ISoLA'16}}
        		{}
        		{\begin{itemize}
        			\item Die RERS Challenge bietet eine Fülle von Problemen mit steigender Komplexität.
        			Die komplexeren Probleme gehen dabei vermutlich über individuelle state-of-the-art
        			Methoden oder Tools hinaus. Charakteristiken von RERS sind sein großer Umfang, welcher
        			nicht nur Quellcode-Analyse, sondern auch (Modell-basierte) Tester und (Test-basierte)
        			Modellierer, insbesondere Free-stylers, anspricht.
        		\end{itemize}}\\
	%\twentyitem{<dates>}{<title>}{<location>}{<description>}
\end{twenty}

\newpage

\makeprofile % Print the sidebar

%----------------------------------------------------------------------------------------
%	 Publications
%----------------------------------------------------------------------------------------

\section{Veröffentlichungen \& Vordrucke}
\begin{twenty} % Environment for a list with descriptions
	\twentyitem
    		{Sep 2018}
		{}
        		{Multi-Modal Route Planning in Road and\\Transit Networks}
        		{\href{https://arxiv.org/abs/1809.05481}{Preprint arXiv}}
        		{}
        		{}\\
	\twentyitem
    		{Mär 2017}
		{}
        		{Minimization of Visibly Pushdown Automata\\Using Partial Max-SAT}
        		{\href{https://link.springer.com/chapter/10.1007/978-3-662-54577-5_27}{TACAS 2017/Springer}}
        		{}
        		{}\\
        	\twentyitem
    		{Mär 2016}
		{}
        		{Minimization of Büchi Automata using Fair Simulation}
        		{\href{https://arxiv.org/abs/1603.01107}{Vordruck arXiv}}
        		{}
        		{}\\
	%\twentyitem{<dates>}{<title>}{<location>}{<description>}
\end{twenty}

%----------------------------------------------------------------------------------------
%	 Projects
%----------------------------------------------------------------------------------------

\section{Projekte}
\begin{twenty} % Environment for a list with descriptions
	\twentyitem
    		{Apr 2018 -}
		{Heute}
        		{Cobweb}
        		{}
        		{journey-planner, multi-modal-search, traffic-networks, shortest-paths}
        		{\begin{itemize}
        			\item Entwicklung eines Backends für multi-modale Routenplannung
        			\item Forschung, Implementierung, Optimierung und Benchmarking von kürzeste-Wege-Algorithmen
        			\item Verarbeitung von OSM, GTFS und Echtzeit-Daten auf riesigen Verkehrsnetzwerken
        		\end{itemize}}\\
	\twentyitem
    		{Dez 2015 -}
		{Okt 2017}
        		{ULTIMATE}
        		{}
        		{program-analysis, model-checking, reachability, termination}
        		{\begin{itemize}
        			\item Entwicklung, Implementierung, Analyse und Benchmarking von Algorithmen um zufällige
        			Finite-Word, Infinite-Word, Nested-Word und Tree-Automaten (FA, NWA, TA) zu Generieren und zu Minimieren
        		\end{itemize}}\\
	\twentyitem
    		{Apr 2016 -}
		{Okt 2017}
        		{Sparkle}
        		{}
        		{api, selenium, mmorpg}
        		{\begin{itemize}
        			\item Erstellung einer robusten API für ein Browser eingebettetes MMORPG, welches web-crawling Mechanismen benutzt.
        		\end{itemize}}\\
	\twentyitem
    		{Jun 2016 -}
		{Apr 2017}
        		{Mem-Eater-Bug}
        		{}
        		{jna, memory-manipulation, code-injection, windows-api}
        		{\begin{itemize}
        			\item Familiarisierung mit Sicherheitstechnologien und Reverse-Engineering
        			\item Erstellung einer API, welche sich in andere Prozesse zur Speichermanipulation und Codeinjection einklingen kann.
        		\end{itemize}}\\
	\twentyitem
    		{Apr 2014 -}
		{Feb 2015}
        		{Antigen}
        		{}
        		{game, real-time-strategy, game-design, scrum}
        		{\begin{itemize}
	        		\item Entwicklung eines Strategy Spiels mit Scrum als Software Development Managing Framework in einem 5-köpfigen Team
        			\item Entwurf und Design einer Echtzeitstrategiespiel-Idee, Erstellung eines detaillierten Game Design Dokuments
        			\item Implementierung des Spiels, Logik und effiziente Kollisionserkennung
        		\end{itemize}}\\
	%\twentyitem{<dates>}{<title>}{<location>}{<description>}
\end{twenty}

\end{document} 
